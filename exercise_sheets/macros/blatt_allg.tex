\def\vorlesung{Coding Reinforcement Learning}
%\def\vorlesung{Math.\ Einf.\ Sachversicherung}
\def\assistent{A. Ferdinand}
\def\assistentEmail{andre.ferdinand@uni-mannheim.de}
\def\semester{HWS 2023}
\def\uni{Universit\"at Mannheim}

\usepackage{amsthm,amsmath,amssymb}
\usepackage{listings}
\usepackage{natbib}
\usepackage{graphicx}
\usepackage{amsthm,amsmath,amssymb}
\usepackage{hyperref}

\usepackage{bbm}
\usepackage{upgreek}
\usepackage{url}
\usepackage[utf8]{inputenc} 
\usepackage[strings]{underscore}

\usepackage[dvipsnames]{xcolor}
\usepackage{csquotes}
 \usepackage{multirow}
 \usepackage{makecell}

\def\TITEL#1{
\gdef\Blatt{#1}
{\small  
Prof.~S. Weißmann/ L. Doering \  \hfill \vorlesung  \hfill \textbf{Herausgabe:} \herausdatum\\
\assistent \hfill \uni \ \semester \hfill \textbf{Vergleich:} \vergleichdatum \\
}
\rule{\linewidth}{1pt}
\begin{center}
  {\Large\bf Blatt #1}\\
\end{center}

\bigskip
}

\def\Abgabe#1#2{
\vfil
\def\Relax{\relax}\def\schriftl{#1\relax}
\ifx\schriftl\Relax\else\bf Abgabe der schriftlichen Aufgaben: #1\\\fi
                            \bf Besprechung aller Aufgaben: #2
        }

%\input /home/schlather/Vorlesung/allg/ma/title.tex
%\input title.tex
\frenchspacing
\allowdisplaybreaks
\textwidth16cm
\topmargin-1cm
\textheight27cm
\voffset-2cm
\hoffset-1cm
\parindent0cm
\sloppypar
\small\normalsize %sonst funktioniert veraenderter Zeilenabstand nicht
\baselineskip12pt %normaler Zeilenabstand...
\renewcommand{\baselinestretch}{1} %...um 1.0 vergroessert
%Allgemeine Neubefehle
\pagestyle{plain}

\long\def\personnel#1{{\sc #1}}
\long\def\personnel#1{}

%\nosol%
%\solnice

\newif\ifloes
\loesfalse
%\loestrue

\newcounter{aufgabe}
\setcounter{aufgabe}{0}
%\def\Aufgabe{\ifnum\value{aufgabe}>0\bigskip\fi\addtocounter{aufgabe}1{\bf Aufgabe \arabic{abschnitt}.\arabic{aufgabe}}\\}


\def\Aufgabe{\ifnum\value{aufgabe}>0\bigskip\fi\addtocounter{aufgabe}1{\bf
    Aufgabe \arabic{abschnitt}.\arabic{aufgabe}}\\\rm}%
\def\Aufgabe{\ifnum\value{aufgabe}>0\bigskip\fi\addtocounter{aufgabe}1{\bf
    Aufgabe \arabic{aufgabe}}\\\rm}%
\def\VorlesungsAufgabe#1{\ifnum\value{aufgabe}>0\bigskip\fi\addtocounter{aufgabe}1{\bf
    Aufgabe \arabic{aufgabe} (Vorlesung, #1)}\\\rm}%
\def\AufgabeX#1{\ifnum\value{aufgabe}>0\bigskip\fi\addtocounter{aufgabe}1{\bf
    Aufgabe \arabic{aufgabe} (schriftl.;%
    \def\x{1}%
    \def\y{#1}%
    \ifx\x\y1 Punkt\else#1 Punkte\fi%
) }\\\rm}
\def\Aufgabe#1{\ifnum\value{aufgabe}>0\bigskip\fi\addtocounter{aufgabe}1{\bf
    Aufgabe \arabic{aufgabe} (%
    \def\x{1}%
    \def\y{#1}%
    \ifx\x\y1 Punkt\else#1 Punkte\fi%
    ;  schriftlich%
) }\\\rm}
\def\AufgabeV#1{\ifnum\value{aufgabe}>0\bigskip\fi\addtocounter{aufgabe}1{\bf
    Aufgabe \arabic{aufgabe} (%
    #1%
    Votieraufgabe%
) }\\\rm}


\newcounter{abschnitt}
\setcounter{abschnitt}{0}
\def\Abschnitt#1{\bigskip\bigskip\begin{center}\addtocounter{abschnitt}1\setcounter{aufgabe}{0}\large Abschnitt \arabic{abschnitt}: Einstieg\end{center}}

\long\def\Hausaufgabe{%\bigskip\begin{center}\Large\bf
                      %Hausaufgabe\end{center}

  \vfil
  {\bf Abgabe der schriftlichen Aufgabe(n) per Mail an \\ \assistentEmail\\  mit
    einem einzigen lauff\"ahigen C-Skript im Anhang, benannt nach Schema
    max.mustermann.Blatt\Blatt.c .}
  \bigskip
 \setcounter{aufgabe}{0}
 \gdef\Aufgabe{\ifnum\value{aufgabe}>0\bigskip\fi\addtocounter{aufgabe}1{\bf Aufgabe H.\arabic{aufgabe}}\\}
}



\def\makeatletter{\catcode`\@=11\relax}
\def\makeatother{\catcode`\@=12\relax}
\makeatletter

\def\Label#1{
  \immediate\write\@auxout{\gdef\string\Ref#1{Blatt \Blatt,
      Aufgabe \arabic{aufgabe}}}
}

% Definition of  \fverbatim{filename}
\newread\ps@stream\newif\ifnot@eof
\newwrite\@unused
\def\xps@typeout#1{{\let\protect\string\immediate\write\@unused{#1}}}
\newif\if@num\@numfalse\countdef\num@z=230\dimendef\num@b=230\dimendef\@breite=231
\long\def\xepsf@aux#1:.:{\if@num\hbox to \num@b{\hfill\numstyle\the\num@z\vspaces\hss}\advance\num@z by 1\fi\hbox to \@breite{\vstyle #1\hss}\\}

\def\numstyle{\it}
\def\vstyle{\tt}
\def\numbering{\@numtrue}
\def\vspaces{\/\ \ }
\newbox\num@box
\def\fverbatim#1{\mbox{}\openin\ps@stream=#1
\ifeof\ps@stream\xps@typeout{Fehler, Datei #1 nicht gefunden}\else
   {\not@eoftrue \chardef\other=12
    \def\do##1{\catcode`##1=\other}\dospecials \catcode`\ =10
    \if@num\num@z1
       \loop
          {\obeyspaces\read\ps@stream to \epsf@tmp}\ifeof\ps@stream\not@eoffalse\fi
          \advance\num@z by 1
       \ifnot@eof\repeat
       \closein\ps@stream
       \setbox\num@box=\hbox{\numstyle\the\num@z\vspaces}
       \num@b\wd\num@box\num@z1
       \@breite\textwidth\advance\@breite by -\num@b
    \fi\not@eoftrue\openin\ps@stream=#1
    \loop
       {\obeyspaces
        \read\ps@stream to \epsf@tmp\global\let\epsf@fileline\epsf@tmp}
       \ifeof\ps@stream\not@eoffalse\else
       \expandafter\xepsf@aux\epsf@fileline:.:%
       \fi
   \ifnot@eof\repeat
   }\closein\ps@stream\fi}%



\makeatother
