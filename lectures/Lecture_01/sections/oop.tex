\section{Introduction to Object Oriented Programming}

\begin{frame}
    \begin{content}{Solid Principles}
        The SOLID principles are a set of five design principles for writing maintainable and flexible software in object-oriented programming. These principles were introduced by Robert C. Martin and are intended to guide developers in creating code that is easy to understand, modify, and extend. Here's an overview of each principle:
        \begin{enumerate}
            \item The \textbf{Single Responsibility Principle (SRP)} states that a class should have only one reason to change, meaning it should have only one responsibility. In other words, a class should encapsulate a single functionality or behavior. This helps in keeping classes focused and makes them less prone to changes when requirements change.
            \item The \textbf{Open/Closed Principle} suggests that software entities (classes, modules, functions, etc.) should be open for extension but closed for modification. This means that you should be able to add new features or behaviors to your code without altering existing code. This is often achieved through the use of abstractions, interfaces, and inheritance.
        \end{enumerate}
    \end{content}
\end{frame}
\begin{frame}
    \begin{content}{Solid Principles}
        \begin{enumerate}
            \setcounter{enumi}{2}
            \item The \textbf{Liskov Substitution Principle (LSP)} emphasizes that objects of a derived class should be able to replace objects of the base class without affecting the correctness of the program. In other words, if a class is a subclass of another class, it should be usable as a substitute for its parent class without causing unexpected behavior.
            \item The \textbf{Interface Segregation Principle (ISP)} suggests that clients should not be forced to depend on interfaces they do not use. This means that large interfaces should be split into smaller, more specific ones so that clients can implement only the methods they actually need. This helps in avoiding unnecessary dependencies and making code more maintainable.
        \end{enumerate}
    \end{content}
\end{frame}
\begin{frame}
    \begin{content}{Solid Principles}
        \begin{enumerate}
            \setcounter{enumi}{4}
            \item The \textbf{Dependency Inversion Principle (DIP)} states that high-level modules should not depend on low-level modules. Both should depend on abstractions. Additionally, abstractions should not depend on details; details should depend on abstractions. This promotes loose coupling between different components of a system and makes it easier to substitute components without affecting the overall architecture.
        \end{enumerate}
    \end{content}
\end{frame}