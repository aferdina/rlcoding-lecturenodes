\section{Sample questions for exam}

\begin{frame}{Question 1}
    \textbf{Topic:} Optimising Reinforcement Learning Models: Callbacks

    \vspace{10pt}

    What is a callback in the context of training reinforcement learning models and how does its implementation work?

    \vspace{20pt}

    \textbf{Answer:} [Insert answer or grading criteria]
\end{frame}

\begin{frame}{Question 2}
    \textbf{Topic:} Neural Networks in Reinforcement Learning
    \vspace{10pt}

    Give two examples of the use of neural networks in connection with reinformcent learning.

    \vspace{20pt}

    \textbf{Answer:} [Insert answer or grading criteria]
\end{frame}

\begin{frame}{Question 3}
    \textbf{Topic:} Implementing Neural Networks
    \vspace{10pt}

    Show schematically how the implementation of neural networks works in the context of the game environment in reinforcement learning. Name two additional parameters that can/must be defined for the creation of a neural network. 

    \vspace{20pt}

    \textbf{Answer:} [Insert answer or grading criteria]
\end{frame}

\begin{frame}{Question 4}
    \textbf{Topic:} Monitoring Training Process: Tensorboard
    \vspace{10pt}

    What is a tensor board and what is it used for in the context of reinforcement learning? 

    \vspace{20pt}

    \textbf{Answer:} [Insert answer or grading criteria]
\end{frame}

\begin{frame}{Question 5}
    \textbf{Topic:} Monitoring Training Process: Tensorboard
    \vspace{10pt}

    What is a tensor board and what is it used for in the context of reinforcement learning? How is the information provided for the tensor board? 

    \vspace{20pt}

    \textbf{Answer:} [Insert answer or grading criteria]
\end{frame}

\begin{frame}{Question 6}
    \textbf{Topic:} Storing Reinforcement Learning Parameter
    \vspace{10pt}

    Name 3 different elements that are stored when training reinforcement learning models. Describe the functions of the individual elements.

    \vspace{20pt}

    \textbf{Answer:} [Insert answer or grading criteria]
\end{frame}

\begin{frame}{Question 7}
    \textbf{Topic:} Modelling Reinforcment Learning: GridWorld
    \vspace{10pt}

    Name 5 components that are needed to model the game GridWorld from the lecture. Also describe the implementation of the individual components.  
    \vspace{20pt}

    \textbf{Answer:} [Insert answer or grading criteria]
\end{frame}

\begin{frame}{Question 8}
    \textbf{Topic:} Modelling Reinforcment Learning: GridWorld
    \vspace{10pt}

    Name 5 components that are needed to model the game GridWorld from the lecture. Also describe the implementation of the individual components.  
    \vspace{20pt}

    \textbf{Answer:} [Insert answer or grading criteria]
\end{frame}

\begin{frame}{Question 9}
    \textbf{Topic:} Implementing Multiarmed Bandit Environments
    \vspace{10pt}

    Create a UML class diagram for the implementation of a multi-armed bandit algorithm for an environment. Name all required methods and attributes. 
    \vspace{20pt}

    \textbf{Answer:} [Insert answer or grading criteria]
\end{frame}

\begin{frame}{Question 10}
    \textbf{Topic:} Implementing Reinforcment Learning Models
    \vspace{10pt}

    Which mathematical model is described by the Python package `Gym`/`Gymnasium`. Name two main components of the package and how they are used. 
    \vspace{20pt}

    \textbf{Answer:} [Insert answer or grading criteria]
\end{frame}

\begin{frame}{Question 11}
    \textbf{Topic:} Updating Neural Networks
    \vspace{10pt}

    Create a flowchart for the training update of neuroyal networks in `torch` in connection with an optimization function.
    \vspace{20pt}

    \textbf{Answer:} [Insert answer or grading criteria]
\end{frame}

\begin{frame}{Question 12}
    \textbf{Topic:} UML Class Diagram: Reinforcement Learning Algorithm
    \vspace{10pt}

    Create a UML class diagram for a Reinformcent Learning algorithm of your choice. 
    \vspace{20pt}

    \textbf{Answer:} [Insert answer or grading criteria]
\end{frame}

\begin{frame}{Question 3}
    \textbf{Topic:} UML Class Diagram: Autonomous Driving
    \vspace{10pt}

    Create a UML class diagram for the game environment of the gym environment from the lecture.
    \vspace{20pt}

    \textbf{Answer:} [Insert answer or grading criteria]
\end{frame}

\begin{frame}{Question 13}
    \textbf{Topic:} Flowchart Diagram for Optimizing Multi-Armed Bandit Algorithms
    \vspace{10pt}

    Create a flowchart for optimizing multi-armed bandit algorithms. 
    \vspace{20pt}

    \textbf{Answer:} [Insert answer or grading criteria]
\end{frame}

\begin{frame}{Question 14}
    \textbf{Topic:} Visualizing Reinforcement Learning Models
    \vspace{10pt}

    What is the Pygame package used for in the context of reinforcement learning and describe two features of the package. 
    \vspace{20pt}

    \textbf{Answer:} [Insert answer or grading criteria]
\end{frame}

\begin{frame}{Question 15}
    \textbf{Topic:} Tabular Methods
    \vspace{10pt}

    Create a class diagram for the implementation of tabular methods from the lecture. What is the structure of the underlying decision rule? 
    \vspace{20pt}

    \textbf{Answer:} [Insert answer or grading criteria]
\end{frame}

\begin{frame}{Question 16-20}
    \textbf{Topic:} Solid Principles
    \vspace{10pt}
    Describe one of the solid principles and explain the principle using the class diagram below. 
    \vspace{20pt}

    \textbf{Answer:} [Insert answer or grading criteria]
\end{frame}

\begin{frame}{Question 21}
    \textbf{Topic:} Multi-Armed Bandits: Metrics
    \vspace{10pt}
    Specify the definition of a metric for the evaluation of multi-armed bandits.
    \vspace{20pt}

    \textbf{Answer:} [Insert answer or grading criteria]
\end{frame}

\begin{frame}{Question 22}
    \textbf{Topic:} Implementing Reinforcement Learning Algorithms
    \vspace{10pt}
    Specify the definition of a metric for the evaluation of multi-armed bandits.
    \vspace{20pt}

    \textbf{Answer:} [Insert answer or grading criteria]
\end{frame}

\begin{frame}{Question 23}
    \textbf{Topic:} Implementing Reinforcement Learning Environments
    
    \vspace{10pt}
    Which methods and attributes need to be defined for a gym environment. Which mathematical concepts are behind the individual attributes. What are the methods used for? 
    \vspace{20pt}

    \textbf{Answer:} [Insert answer or grading criteria]
\end{frame}

\begin{frame}{Question 24}
    \textbf{Topic:} Implementing Reinforcement Learning Environments

    \vspace{10pt}
    Which methods and attributes need to be defined for a gym environment. Which mathematical concepts are behind the individual attributes. What are the methods used for? 
    \vspace{20pt}

    \textbf{Answer:} [Insert answer or grading criteria]
\end{frame}

\begin{frame}{Question 25}
    \textbf{Topic:} Wrapper Classes and Reinforcement Learning Environments

    \vspace{10pt}
    How does a wrapper class work and what is it used for in the context of reinforcement learning environments?   
    \vspace{20pt}

    \textbf{Answer:} [Insert answer or grading criteria]
\end{frame}

\begin{frame}{Question 26}
    \textbf{Topic:} Hyperameter Tuning

    \vspace{10pt}
    Name three building blocks for hyperparameter optimization in the context of reinformcent learning.  
    \vspace{20pt}

    \textbf{Answer:} [Insert answer or grading criteria]
\end{frame}
\begin{frame}{Question 27}
    \textbf{Topic:} Policy Iteration Algorithm: Flow Diagram

    \vspace{10pt}
    Create a flowchart for the implementation of the algorithm. 
    \vspace{20pt}

    \textbf{Answer:} [Insert answer or grading criteria]
\end{frame}

\begin{frame}{Question 28}
    \textbf{Topic:} Dynyamic Programming: Flow Diagram

    \vspace{10pt}
    Create a flowchart for the implementation of the algorithm. 
    \vspace{20pt}

    \textbf{Answer:} [Insert answer or grading criteria]
\end{frame}

