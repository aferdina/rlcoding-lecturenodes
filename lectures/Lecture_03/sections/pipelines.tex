\section{Pipelines}


\begin{frame}
    \begin{columns}
        \begin{column}{0.5\textwidth}
            \begin{figure}    
                \includegraphics[width=1.0\textwidth]{./images/pipeline01.png}
                \caption{Pipeline \href{https://gitlab.com/aferdina/MultiArmedBandits}{multiarmedbandits}, 27.09.23} 
            \end{figure}
        \end{column}
        \begin{column}{0.5\textwidth}
            \begin{figure}    
                \includegraphics[width=1.0\textwidth]{./images/pipeline02.png}
                \caption{Pipeline \href{https://gitlab.com/aferdina/MultiArmedBandits}{multiarmedbandits}, 27.09.23}
            \end{figure}
          \end{column}
        \end{columns}
\end{frame}
\begin{frame}
    \begin{content}{Pipelines}
        
In the context of software development, a pipeline typically refers to a \textbf{sequence of automated steps} or stages that code and other assets go through as they progress from development to deployment. These pipelines are designed to \textbf{streamline the development} and deployment process, \textbf{improve code quality}, and ensure that software changes are thoroughly \textbf{tested} before they reach production environments. The primary types of pipelines you'll encounter in this course is a \textbf{Continous Integration (CI) pipeline}:
\begin{itemize}
    \item CI pipelines focus on the \textbf{integration of code} changes into a shared repository. Developers frequently \textbf{merge their code into a common codebase}, and the \textbf{CI pipeline automatically builds, tests, and validates} these changes.
    \item The CI pipeline helps identify integration issues early in the development process and ensures that the codebase remains in a stable state.
\end{itemize}
    \end{content}
\end{frame}