\section{Testing}
\begin{frame}
    \begin{content}{Components of testing}
        \begin{itemize}
            \item Code coverage
            \item Code quality
            \item Code style
        \end{itemize}
    \end{content}
\end{frame}
\begin{frame}
    \begin{content}{Code coverage}
        Code coverage is a metric used in software testing to measure how thoroughly your tests exercise your codebase. It provides insights into which parts of your code have been executed by your tests and which parts remain untested. The primary goal of code coverage analysis is to identify areas of your code that may lack proper testing and could potentially contain defects or bugs.
    \end{content}
\end{frame}
\begin{frame}
    \begin{content}{Unit Tests}
        A unit test is a type of software testing where individual units or components of a software application are tested in isolation from the rest of the system to ensure that they function correctly. The term `unit' in unit testing refers to the smallest testable part of a software program, typically a function, method, or class. The primary purpose of unit testing is to validate that each unit of code behaves as expected and produces the correct output for a given set of inputs.
    \end{content}
\end{frame}